\documentclass[a4paper]{article}
\usepackage[utf8]{inputenc}
\usepackage[T1]{fontenc}
\usepackage[slovene]{babel}
\usepackage{lmodern} 
\usepackage{hyperref}
\usepackage{blindtext}
\usepackage{amsmath}  % razna okolja za poravnane enačbe ipd.
\usepackage{amsthm}   % definicije okolij za izreke, definicije, ...
\usepackage{amssymb}  % dodatni matematični simboli
\usepackage{listings} % okolje za Python kodo

\title{\textit{Grid-peeling}}
\author{Gašper Pust, Mitja Mandić}

\begin{document}
\lstset{language=Python, breaklines=true, columns=fullflexible, basicstyle=\ttfamily}
\begin{titlepage}
 \maketitle
% \thispagestyle{empty}
\end{titlepage}
%
%\pagebreak

\section{Predstavitev problema}
V projektu si bomo podrobneje ogledali konveksne ovojnice $m \times n$ mreže. Konveksna ovojnica množice je najmanjša konveksna množica, ki vsebuje dano množico.
Najlažje si jo predstavljamo tako, kot da bi okoli elementov množice napeli elastiko - kar elastika obkroži, je konveksna ovojnica. Lupljenje konveksnih ovojnic mreže,
oziroma angleško \textit{grid - peeling} je proces, ko iz mreže iterativno odstranjujemo konveksne ovojnice. S simboli lahko to zapišemo takole:
$ P_{0} = G_{n,m} = \{1,\ldots, n\} \times \{1, \ldots, m\}$. Naj bo $C_{i} = \mathcal{C}\mathcal{H}(P_{i-1}) \text{ za } i = 1, \ldots$. $V_{i}$ naj bo množica vozlišč $C_{i}$
- kot vozlišče razumemo točko, ki je na vogalu mreže (torej za katero bi zataknili elastiko). Naj bo sedaj $P_{i} = P_{i-1} \setminus V_{i}$. Začnemo torej z $n \times m$ mrežo 
in iterativno lupimo konveksne ovojnice, dokler ne odstranimo vseh točk.

V projektni nalogi bova s pomočjo simulacij opazovala v literaturi navedene številke za $n \times n$ mrežo - teorija napoveduje $\theta(n ^ \frac{4}{3})$ ovojnic.
Za $n \times m$ mrežo v literaturi ni navedenih podatkov, zanimala naju bo morebitna povezava. Simulacije bova izvedla tudi za točke na neenakomerni mreži.

Po izvedenem eksperimentalnem delu, bomo rezultate analizirali in jih primerjali z rezultati iz literature. Zanimalo nas bo, kako drugačno je število ovojnic na $m \times n$
mreži v primerjavi s simetrično.

\section{Orodja in algoritmi}
\subsection{Jarvisov obhod}
Jarvisov obhod (angl. \textit{Jarvis March}) ali algoritem zavijanja darila je postopek, ki dani množici točk poišče konveksno ovojnico v eni ali več dimenzijah (osredotočili se 
bomo na dve dimenziji). Algoritem se imenuje po R.A. Jarvisu, ki ga je objavil leta 1973. Časovna zahtevnost algoritma je $O(nh)$, kjer $n$ predstavlja število vseh točk, $h$ pa 
število točk, ki ležijo na konveksni ovojnici. V najslabšem primeru, ko so vse podane točke tudi elementi konveksne ovojnice, torej v primeru $h = n$, je njegova časovna zahtevnost 
$O(n^2)$. Jarvisov obhod se največkrat uporablja za majhne $n$ ali pa v primeru, ko pričakujemo, da bo $h$ zelo majhen glede na $n$.

\begin{lstlisting}
def jarvis_march(points):
    # find the leftmost point
    a =  min(points, key = lambda point: point.x)
    index = points.index(a)
    
    # selection sort
    l = index
    result = []
    result.append(a)
    while (True):
        q = (l + 1) % len(points)
        for i in range(len(points)):
            if i == l:
                continue
            # find the greatest left turn
            # in case of collinearity, consider the farthest point
            d = direction(points[l], points[i], points[q])
            if d > 0 or (d == 0 and distance_sq(points[i], points[l]) > distance_sq(points[q], points[l])):
                q = i
        l = q
        if l == index:
            break
        result.append(points[q])

    return result
\end{lstlisting}

%https://algorithmtutor.com/Computational-Geometry/Convex-Hull-Algorithms-Jarvis-s-March/

\subsection{Grahamov pregled}
Alternativa prejšnjemu algoritmu je tako imenovani Grahamov pregled (angl. \textit{Graham's scan}). Algoritem se imenuje po Ronaldu Grahamu, ki ga je objavil leta 1972. V primerjavi z Jarvisovim obhodom je Grahamov pregled hitrejši, saj ima časovno zahtevnost $O(n \log n)$.

\begin{lstlisting}
def convex_hull_graham(points):
    TURN_LEFT, TURN_RIGHT, TURN_NONE = (1, - 1, 0)

    def cmp(a, b):
        return (a > b) - (a < b)

    def turn(p, q, r):
        return cmp((q[0] - p[0])*(r[1] - p[1]) - (r[0] - p[0])*(q[1] - p[1]), 0)

    def _keep_left(hull, r):
        while len(hull) > 1 and turn(hull[-2], hull[-1], r) != TURN_LEFT:
            hull.pop()
        if not len(hull) or hull[ - 1] != r:
            hull.append(r)
        return hull

    points = sorted(points)
    l = reduce(_keep_left, points, [])
    u = reduce(_keep_left, reversed(points), [])
    return l.extend(u[i] for i in range(1, len(u) - 1)) or l
\end{lstlisting}

%https://gist.github.com/arthur-e/5cf52962341310f438e96c1f3c3398b8

%lahko tudi algoritem iz 
%https://algorithmtutor.com/Computational-Geometry/Convex-Hull-Algorithms-Graham-Scan/



\section{Rezultati}

\section{Zaključek}


\end{document}