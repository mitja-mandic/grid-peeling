\documentclass[a4paper]{article}
\usepackage[utf8]{inputenc}
\usepackage[T1]{fontenc}
\usepackage[slovene]{babel}
\usepackage{lmodern} 
\usepackage{hyperref}
\usepackage{blindtext}
\usepackage{amsmath}  % razna okolja za poravnane enačbe ipd.
\usepackage{amsthm}   % definicije okolij za izreke, definicije, ...
\usepackage{amssymb}  % dodatni matematični simboli

\title{\textit{Grid-peeling} - kratek opis projekta}
\author{Gašper Pust, Mitja Mandić}

\begin{document}
\begin{titlepage}
 \maketitle
% \thispagestyle{empty}
\end{titlepage}
%
%\pagebreak

\section{Opis problema}
V projektu si bomo podrobneje ogledali konveksne ovojnice $m \times n$ mreže. Konveksna ovojnica množice je najmanjša konveksna množica, ki vsebuje dano množico.
Najlažje si jo predstavljamo tako, kot da bi okoli elementov množice napeli elastiko - kar elastika obkroži, je konveksna ovojnica. Lupljenje konveksnih ovojnic mreže,
oziroma angleško \textit{grid - peeling} je proces, ko iz mreže iterativno odstranjujemo konveksne ovojnice. S simboli lahko to zapišemo takole:
$ P_{0} = G_{n,m} = \{1,\ldots, n\} \times \{1, \ldots, m\}$. Naj bo $C_{i} = \mathcal{C}\mathcal{H}(P_{i-1}) \text{ za } i = 1, \ldots$. $V_{i}$ naj bo množica vozlišč $C_{i}$
- kot vozlišče razumemo točko, ki je na vogalu mreže (torej za katero bi zataknili elastiko). Naj bo sedaj $P_{i} = P_{i-1} \setminus V_{i}$. Začnemo torej z $n \times m$ mrežo 
in iterativno lupimo konveksne ovojnice, dokler ne odstranimo vseh točk.

V projektni nalogi bova s pomočjo simulacij opazovala v literaturi navedene številke za $n \times n$ mrežo - teorija napoveduje $\theta(n ^ \frac{4}{3})$ ovojnic.
Za $n \times m$ mrežo v literaturi ni navedenih podatkov, zanimala naju bo morebitna povezava. Simulacije bova izvedla tudi za točke na neenakomerni mreži.

\section{Orodja in algoritmi}
Za iskanje ovojnic bova v Pythonu implementirala Jarvis-march (gift - wrapping) algoritem ali Graham-scan. Prvi je enostavnejši, vendar nekoliko počasnejši.
Jarvisov algoritem na vsakem koraku pregleda vse točke, ki niso v ovojnici in vanjo doda tiste, ki so najbolj levo in najdlje od trenutne točke. Podrobneje ga
bomo predstavili v zaključnem poročilu.

Po izvedenem eksperimentalnem delu, bomo rezultate analizirali in jih primerjali z rezultati iz literature. Zanimalo nas bo, kako drugačno je število ovojnic na $m \times n$
mreži v primerjavi s simetrično.



\end{document}}